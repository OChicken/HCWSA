%!TEX root = hcwsa.tex

\chapter{初来乍到}\label{chap:初来乍到}

  本章主要分为六部分:

  \begin{enumerate}
    \item 入境前要办的事情有:银行开户(\ref{sec:银行}),申请私立医保用于签证(B.2.1节),寻找住房(B.3章)。这些事情在RWTH发Zulassung后就可以着手办理。
    \item B.4章涉及申请签证和入境前往亚琛。
    \item B.5章涉及对大学市政厅、SuperC、外管局三个重要办事点的介绍。
    \item 入境后两周内要办的事情有:购买电话卡和办理宽带,户口登记/Anmelden,银行账户激活,申请公立医保用于入学注册,这些都在B.6章谈及。
    \item B.7章涉及入学注册、初步接触学校的学生管理系统以及使用其邮箱。
    \item B.8章涉及延签。
  \end{enumerate}
  
  Anmelden作为一个关键概念须在Part B引言部分提出。Anmelden是德语中 “(户口)登记”1 的意思,简称An。类似的一个词是Abmelden,“(户口)退登记”。在德国境内居住的德国人和外国人都有登记户口的义务。Anmelden手续是凭住房合同到当地市政厅办理的,完成后您将现场拿到一张纸质证明,证明您在当地有固定住所。有了Anmelden的纸质证明,您才可以办理银行业务、消费、公立医保申请、入学注册等事情。

  为使概念明晰,我们把银行的开户以及激活都放在B.1章以区分在亚琛的两个银行——德意志银行和亚琛储蓄银行的业务办理的异同,虽然激活是入境后办理的。同理,我们把私立医保和公立医保的申请以及缴费等相关事宜放在B.2章以区分两类医保的异同,虽然公立医保的申请是Anmelden之后办理的。涉及到的相关概念较多,因此各单独作为一章。

  特地在B.5章专门讲述三个办事点,是因为历史和地理原因,导致在描述这些地点的时候往往采用不同的词,以致给新生造成混乱。特此澄清,以便于给后三章做铺垫。

  综上,我们都尽可能地在初次出现某些概念的时候对相关的有含混的用词一次性予以澄清,因此时间线未必非常明朗,特此在Part B开头强调这一点。

\section{银行}\label{sec:银行}

  外籍学生都必须使用德国境内的银行进行消费,它是申请德国签证必需的材料之一;此外,银行的选择涉及到在当地消费的便捷度。

  为使概念明晰,我们把银行的\textsbf{开户}和\textsbf{激活}都放在这里介绍,尽管激活是入境后才需要去办的事情。

  \subsection{概念澄清}
 
    目前亚琛有两家较大的银行支持在国内办理\textsbf{开户/Account Opening},分别是\href{https://www.deutsche-bank.de/pk.html}{\textsbf{德意志银行/德银/Deutsche Bank/DB}}和\href{https://www.sparkasse-aachen.de/de/home.html}{\textsbf{亚琛储蓄银行/Sparkasse Aachen/Sparkasse}}。后文用德银和Sparkasse指代。

    用于\textbf{申请德国签证}所开通的德国境内的银行账户是指\textsbf{冻结账户/Blocked Account/Sperrkonto},存入的资金叫\textsbf{冻结资金/Blocked Balance/Sperrbetrag}。词缀\textsbf{冻结/Blocked/Sperr-}指的是开户人在前往开户行所在国(德国)之前\textbf{不得使用}该账户内的资金,这是各国银行出于涉外金融安全的考虑而做的举措。类似字样也出现于签证相关的文件里的 “冻结” 和 “解冻”。开户会涉及到\textbf{手续费} (含信件运费),通常是150欧左右。

    用于\textbf{在德国当地消费}的银行账户称为\textsbf{借记账户/Debit Account/Girokonto}。在到达了亚琛、有了固定地址、办理了\textsbf{Anmelden的证明}以后,即可办理Sperrkonto和Girokonto的激活手续。

    用于日常消费 (如支付房租) 的银行卡是指Girokonto的\textsbf{借记卡/储蓄卡/Debit Card/Girocard}及其IBAN。Sperrkonto是不发卡的。

    每一个留德学生的名下都有Sperrkonto和Girokonto。这两个账户的关系是:\textbf{Sperrkonto每月自动把一定数额的欧元(853欧)汇入到Girokonto}。关于这两个账户跟德银和Sparkasse的区别见Table.1。

    \textsbf{冻结账户/Sperrkonto}是被提款的,因此它也被称为\textsbf{限制提款账户} (这出现在签证文件的用词里)。\textsbf{冻结资金}也称\textsbf{自保金},每月提款额度不得超过一定金额,如853欧。所规定的这853欧是最基本的生活保障,是为了保障留德学生有足够的资金维持在当地一年共计10,236欧的生活费用\footnote{当然,您还可以通过其他方式获得更多的生活费。所以853欧并不是固定的,在开户表格上可以自行决定。}。这个数额是德国政府基于在德大学生的消费调查而规定的,相对稳定,不随利率波动\footnote{2018年的自保金数额是8,640欧,这个数额已沿用多年,而2020年的数额则是10,236欧。}。

    Fig.1是两个银行的卡的样例。欧盟的银行系统里对\textsbf{账户号/Konto-Nr/IBAN}和\textsbf{卡号/Karten-Nr}是有区分的。\textsbf{Karten-Nr}的具体格式取决于银行。这种设计是为了防止卡遗失\footnote{挂失后新发的卡的Karten-Nr跟旧卡的Karten-Nr不一样,但他们的IBAN是一样的。}。

    银行系统有\textsbf{借记卡/储蓄卡/Debit Card/Girocard} (不可透支) 和\textsbf{贷记卡/信用卡/Kreditkarte} (可透支) 两种类型的账户的区分。借记卡和贷记卡是书面语,储蓄卡和信用卡是口语。由于货币的结算方式不同,银行规定,借记卡只能用于发卡行的境内消费,而贷记卡则可以用于国际消费,以特定的结算方式计入到发卡行所在国的持卡人的借记卡上。\textsbf{银联/UnionPay}、\textsbf{VISA}、\textsbf{MasterCard/万事达}是不同贷记卡所采用的不同的结算方式。Sparkasse在信用卡的业务上推行了不少福利,在办理Girokonto激活业务的同时会推荐办理其贷记卡。

    \begin{table}[h]
      \centering
      \caption{title}
      \begin{tabular}{|c|c|c|}
      \hline
      &Sperrkonto&Girokonto\\
      \hline
      德银&\multicolumn{2}{c|}{同一个IBAN}\\
      \hline
      Sparkasse&IBAN for Sperrkonto&IBAN for Girokonto\\
      \hline
      \end{tabular}
    \end{table}

    下面分别讲述德银和Sparkasse的开户和激活流程。

  \subsection{德意志银行}

    德意志银行的开户流程如这里所示:

    \begin{enumerate}
      \item 从官网下载申请表格并且填写。默认853欧的每月提款额度可以填写其他数额。
      \item 前往德银驻北/上/广分行办理相关材料,并将拿到个人编号通知单。
      \item 根据个人编号通知单的信息汇款。该通知单上有两个收款账号:一个是德银总部的负责相关业务部门的IBAN,汇款的8640欧是自保金;另一个是 德银驻华分行的工作人员的账号,汇的是手续费。
      \item 先后收到来自德银驻北/上/广分行的两份纸质信函,分别是存款证明和开户证明/Kontoeröffnung。后者有您的IBAN。
    \end{enumerate}

    至此,德银的开户流程办理完毕。下面是激活流程:

    \begin{enumerate}
      \setcounter{enumi}{4}
      \item Anmelden,得到相关的证明,上面有您在德国的长期住址。
      \item 在官网->Activate a blocked account下载表格填写,打印,签字。街头的打印店有unicopy。
      \item 邮寄给汉堡办事处,网上搜索\textsbf{Deutsche Post}可知邮筒的位置。不必去\textsbf{德银驻亚琛支行/Deutsche Bank Filiale Aachen}。
      \item 大约一周以后,您会收到来自德银的若干封\textbf{纸质信件},包括但不限于网银密码、Girocard和卡密码、photoTAN (德国版的网银支付二维码),等等。
    \end{enumerate}

  \subsection{Sparkasse Aachen}

    Sparkasse Aachen的办理流程如这里所示,实际上发邮件即可:

    \begin{enumerate}
      \item 根据如下模板发邮件给 896-visa@sparkasse-aachen.de 告知银行的工作人员说您要开通Sperrkonto。附件为\textbf{大学录取通知书}和\textbf{护照}。
    \end{enumerate}

    \noindent\fbox{
      \begin{minipage}{\textwidth}
        Betreff: Sperrkonto/ blocked account

        Sehr geehrte Damen, sehr geehrte Herren,

        Dear Sir or Madam,

        \vskip1em

        Als zukunftige Studentin einer Hochschule in Aachen mochte ich bei der Sparkasse Aachen ein Sperrkonto einrichten.

        As a future student at one of the universities in Aachen, I would like to create a blocked account at Sparkasse Aachen.

        \vskip1em

        Die Bestatigung liber den Studienplatz der Hochschule und eine Passkopie habe ich als Emailanhang beigefiigt.

        The confirmation of acceptance at the university and the passport copy is attached.

        \vskip1em

        Viele GruBe

        Kind regards

        Vorname Nachname
      \end{minipage}
    }

    \begin{enumerate}
      \setcounter{enumi}{1}
      \item 工作人员回复,让您汇款给Sparkasse的账户,以及提供您的地址和电话等资料。 
      \item 收到存款证明和开户证明/Kontoeröffnung这两个文件的纸质信函和Email。
    \end{enumerate}

    至此,Sparkasse的开户流程办理完毕。下面是激活流程:

    \begin{enumerate}
      \setcounter{enumi}{3}
      \item Anmelden,得到相关的证明,上面有您在德国的长期住址。
      \item 打电话约Termin,届时带上护照和Anmelden的证明前往Sparkasse Aachen Filiale办理Sperrkonto的激活以及Girokonto的开户和激活。在Girokonto的开户表格上确定每月提款额度。
      \item 收到网银密码、Girocard和卡密码。
    \end{enumerate}

  \subsection{两家银行的对比}

    经过上述步骤即可在银行或者通过ATM机自助取款了\footnote{Sperrkonto 激活激活之前也可以用国内的\textbf{银联卡}在 ATM 机取款,是根据实时汇率从人民币扣款的。}。Sparkasse Aachen在全亚琛有大大小小的ATM机以及银行服务点,取钱和办理各种银行服务较为方便。而德银的冻结资金办理手续比较繁琐,且其ATM机在亚琛的分布相对较少。

    Sparkasse Aachen提供的Girokonto服务比较多样,如\href{https://www.sparkasse-aachen.de/de/home/privatkunden/girokonto/s-pool.html}{\textsbf{S-POOL Konto}}。这是Sparkasse Aachen专门为18-30岁的学生准备的青年账户。只需要出示有效的学生证明,办理S-POOL Konto的账户时即可享受优惠的账户管理费。办理S-POOL账户时,还可以选择免费申请一张Mastercard,对于买机票住酒店等大额消费来说非常的方便。此外,S-POOL 账户还和亚琛的诸多商家有合作,付款时使用S-POOL卡能够享受优惠。

    因此相较来讲,Sparkasse Aachen是在亚琛学习和生活的首选。

\section{医疗保险}\label{sec:医疗保险}

  在德国大学生必须购买\textsbf{医疗保险/Krankenversicherung}。医疗保险公司有\textsbf{公立保险/法定医疗保险}和\textsbf{私立保险}之分,也就是我们俗称的\textsbf{公保}和\textsbf{私保}。\textsbf{私保}是用于递签的保险,它区别于签证文件上提及的\textsbf{法定医疗保险/公保}。用于递签的只能是私保而不是公保,因为递签之时您还没有在德国有固定地址;当您有了在德固定地址以后,才可以申请公保。

  \subsection{私保申请和缴费}\label{subsec:私保申请和缴费}

    最终完成学校的申请须提交私保的开通证明。私保的选取大同小异,跟公保公司AOK合作的私保是Klemmer。按邮件的指引签署扣款协议,钱就从您的在国内银行的贷记卡中扣除。这就完成了``入境德国180天有效、直到法定医疗保险生效''\footnote{\href{https://china.diplo.de/blob/1341652/cd1526772cd3d64a3f122f07d908a16b/pdf-merkblatt-natvisum-studium-data.pdf}{德国驻华使领馆->服务->签证和入境->长期停留->留学签证申请须知}} 这段时间内的医保扣款。

  \subsection{公保申请}\label{subsec:公保申请}

    一般的留德学生都会选择公保,公保的保险公司也有很多,\href{https://www.aok.de/}{AOK} (有\href{http://www.aok-cn.com/}{中文网站},位于Ursulinerstraße的门店如Fig.2所示),\href{https://www.tk.de/techniker}{TK} (在亚琛的门店较AOK更多),DAK等等,每个月的保险费在90欧左右。持语言签证的学生只能选择私保,每月保险费28欧起。来德国高校的交换生也推荐私保,因为此时私保的价格相对于公保要更便宜。

    办理公保所需的材料包括:护照、银行帐号、录取通知书,具体如B.6.4节所述。在高校注册入学时,需要出示医疗保险证明,可以向医疗保险公司索取。公保的缴费是每月从IBAN中自动扣除的。

    若您选择Klemmer,则当您有了Anmelden的证明、并按邮件指引进一步更新了住址的同时,该公司就帮您同时申请了AOK公保;当然您也可以选择仅扣款,而公保自己找心仪的公保公司。

\section{住房}\label{sec:住房}

  亚琛在9月到11月期间,房源会变得异常紧俏。为了在初到亚琛之际能有一个落脚点,建议您尽可能通过多种渠道去寻找房源,先找个可以Anmelden的房子落脚,等找房高峰过去并初步熟悉当地环境以后,再慢慢寻觅符合自己需求的房源。

  \subsection{住房形式}\label{subsec:住房形式}

    德国高校的\textsbf{大学生服务中心/Studentenwerke}为学生提供了一系列的服务,从大学的租房网页上可以概括性地了解相关信息,如水电暖网以及房屋内景等。

    其次有由房屋公司提供的\textsbf{公寓}或者房东是当地人的\textsbf{私房}。``公寓''在国内语境不难理解,而``私房''则往往意味着会有同居的室友,这种情形也描述为``您跟您的室友\textsbf{WG}一个房子''。因为在德国,很常见的一类房子是长这样的:它是某个小楼的某一层中的其中一间\textsbf{Wohnung},进去后是2~3个\textsbf{小房间/Zimmer}租给不同的租客,公用厨房 (厨具最好用自己的;若自己当前租的房子是zwischen的话就用二房东的)、厕所和冲凉房。洗衣机可以是在冲凉房内、地下室里,或者公寓附近。

    一个房间只允许单人居住,除非能提供结婚证才可允许夫妻同居。

  \subsection{租房信息}\label{subsec:租房信息}

    在亚琛寻找房源主要有以下渠道,通常前两个是最稳妥最常用的:

    \textbullet 学生公寓:所有宿舍的申请入口\href{https://bewerberportal.stw.rwth-aachen.de/app.php/de/}{汇总}。申请学生宿舍一般会有一段较长的排队等待时间。

    \textbullet 中文社群:微博、QQ群、微信群。作为一座大学城,亚琛的房源流动性很强,也由此有很多本地找房微信群。微信群的入口往往是通过自己临时住房的中国籍的二房东获取的。

    \textbullet 德语网站:\href{http://www.wg-gesucht.de}{WG-Gesucht.de}是主要的找房网站,房源多且没有中介费;\href{http://www.studenten-wg.de}{HousingAnywhere}房源较多。

    \textbullet 私人学生宿舍:这个\href{http://www.stwkhg.de/}{教会宿舍}有多栋学生宿舍;\href{http://www.esg.rwth-aachen.de}{ESG Aachen}位置优秀,离学校仅大约数分钟步行路程。\href{http://www.t39.rwth-aachen.de/}{Templergraben 39}是学生自治宿舍。

    \textbullet 布告栏:Mensa,Audimax,Mogam等自习室的信息栏。

  \subsection{租房合同和注意事项}\label{subsec:租房合同和注意事项}

    在德国,签订住房合同是一件很慎重的事情,签字之前务必仔细阅读住房合同。一份合同的有效期通常是一年,为双方省事,一年满后默认续签;需要特别注意的是租房合同中与扣取押金及解除合同后押金返还有关的条目,不清楚的需要向房东详细询问。

    德国租房分冷租 (Kaltmiete) 和暖租 (Warmmiete) 两种。暖租即房租中包括了水电暖、网费、管理费等费用,冷租则需要去水电公司自行办理,水电公司可以通过Check24寻找最优惠的。若不包网费,得自行去运营商门店办理宽带 (B.6.1节)。

    有了德银的IBAN码,是可以在没有收到银行卡的情况下签订房租的自动扣款合同的。

    敲定以上事宜后,房东会草拟一份合同抄送给您,但这份合同并未经您签字。在入住和签订合同之前,房东会带您在房间里查看一圈,清点家具并检查房间的损坏程度。签了字,并按要求付了\textsbf{押金} (通常为房租月租的1~3倍) 之后,您就是房间的主人了:倘若以后有物品损坏,会被扣押金。

    房东交给您的钥匙一定要保管好,因为在德国,很可能一层楼都用同一把钥匙开门。一旦钥匙丢了,房东很可能会把一层楼的锁都换掉。这个风险不可能通过私自配钥匙避免,因为在德国配钥匙需要房东和警察局的双重证明到专门的店里去配才行。

    如果您要换房,可以以提前找好续租者/下一任房客/Nachmieter的方式提前结束租房合同。否则,须提前三个月提前通知房东以解约。

    拿到合同并An了之后记得在自己的门牌上贴上自己的名字,或要求房东更换公寓楼下的住户列表,确保快递员和信使找得到您的名字。同时,\textbf{记得到RWTHonline更新地址},以免学校寄给您的纸质信件丢失。

  \subsection{租房广告里的常见名词之定义}\label{subsec:租房广告里的常见名词之定义}

    鉴于微信群是目前最有效的房源,我们特地把租房广告里常见的名词缩写加入在此,因为这些名词经常出现在租房广告里。

    \textsbf{An}: Anmelden的缩写。微信群的租房广告中常把Anmelden简称An。

    \textsbf{房东/房主}:房屋的产权人,合同中的乙方。

    \textsbf{二房东/Untervermieter/出房人}:往往是在这里就读了一段时日或者是即将毕业的学生。ta跟房东签订了合同并有该房子的Anmelden证明。

    \textsbf{租客/续租者/下一任房客/Nachmieter}:就是读者,”您”。

    \textsbf{可An房}:二房东须解除ta自己的租房合同并去市政厅/Bürgerservice (B.5.1节\&B.6.2节) 退注册,ta因为毕业或者是找到了更好的房子需要搬家故把旧房转出。同时您要跟房东签署您自己的租房合同并去Bürgerservice拿Anmelden证明。

    \textsbf{Zwischen房/不可An房/临时住房}:二房东不解除合同,不去Bürgerservice退注册,ta仅仅是实习或者假期回国导致房子闲置所以出房,中间的这一两个月给租客来住,从而填补这一两个月的租金。租客也因此不必跟房东签合同,但会跟二房东签订一份形式上的合同 (私下商定)。

    \textsbf{WG}:即Wohngemeinschaft,合租。

    \textsbf{Wohnung \& Zimmer}:Zimmer $\in$ Wohnung。譬如在租房广告里常见的表述:5 Zimmer Wohnung。

    \textsbf{Nach}:即家具转出。通常房子的家具都能留下给下一个租客,出钱买或者不出钱买只是租客跟二房东私下商定;若租客是去一些公司管理的公寓直接找房或者是排到学校公寓的话,这个房子往往是空房,没有家具,得自己买,比较麻烦。

    \textsbf{押金/Kaution}:搬入时需交纳,通常为房租的1-3倍,需要特别注意的是租房合同中与扣取押金及解除合同后押金返还有关的条目,不清楚的需要向房东详细询问。

    \textsbf{定金}:跟押金经常混用。二房东发布广告后往往有不止一个潜在租客去加好友申请看房,但他们都未必会当即决定要这个房子。通常最先付定金的那个潜在租客往往就是该房子的下一任租客。定金通常不超过200欧,逼近四位数的定金就要慎重。

  \subsection{出境前寻找住房的注意事项}\label{subsec:出境前寻找住房的注意事项}

    因为您此时在国内而房东在国外,所以这里列出在出境前寻找和确定住房需要注意的点:

    \begin{enumerate}
      \item 最好是选择可以Anmelden的房子,可以是大学学生宿舍、当地公寓或私房,这样您能在到达亚琛的一个月内签订合同,并进而办理Anmelden手续、银行账户激活和公保申请(B.6章)。如果房子是临时住房,也一定要尽快确定可以An的房子,否则生活会有很多不便利。
      \item 大学学生公寓最为方便省事,因此它是大部分学生的首选,这也意味着申请人数众多。只要提前足够长的时间申请,就有很好的机会拿到学生宿舍的房间。因此通常申请大学的同时就申请学生宿舍。
      \item 对于在微博或者微信群里找到的私房 (二房东通常是中国人),可以通过视频实现在线看房,技术上不是难度。如果双方谈妥,可以先交定金,这样一到亚琛就可以办理签合同事宜。
      \item 须关心住所跟学校、学院、上课地点的相对位置,以便节省通勤时间。
    \end{enumerate}


\section{申请签证及入境}\label{sec:申请签证及入境}

  有了前三章的铺垫,申请签证就水到渠成。从国内到亚琛市通常包含三步:购买机票入境,从入境城市乘坐列车到达亚琛,从车站乘公交到住所。

  \subsection{签证申请}\label{subsec:签证申请}

    签证前通常要完成的事情包括APS面谈,学校申请,(CSC申请),银行开户和私立医保申请。因为已经有了APS证书,所以签证是免面试的,所要做的是到德国驻华使领馆提交签证申请表即可。递签的流程和所需材料皆可在德国驻华使领馆->服务->签证和入境->长期停留上找到。百度经验等平台上都有不错的经验帖,这里就不赘述了。

    注意到所填的 “预计进入申根区日期”,这意味着您在递签之前要确定好自己的机票。关于订机票的事宜详见下一小节。

  \subsection{购买机票入境}\label{subsec:购买机票入境}

    从携程、飞猪等中间商,或者从航空公司网站 (汉莎航空是德国最大的航空公司) 直接查询并购买机票。通常的目的地是杜塞尔多夫或者法兰克福。两个城市都有其优劣:杜塞尔多夫跟亚琛都在北威州,火车频次多,几乎不用转车,但航班班次相对较少;法兰克福是德国的经济重镇,航班班次较多,但入境查个人物品比较严格。

    入境时间跟开学时间不要相差太短但也不必太长。大约两周左右即可搞定B.6章和B.7章的所有事情。

    初次入境会比较多地考虑行李限额。部分航空公司对留学生有行李优惠,能以经济舱的票价享受托运大件行李 (包括托运行李的个数以及总重) 的权利。具体须向航空公司咨询。

    特别强调通讯网络的重要性,尤其是当您踏上异国的土地的时候,手机没有网,是寸步难行的。建议在国内淘宝购买欧洲旅行专用的电话卡 (通常一个月失效,但已足够时间在德国买一个新的) 。

  \subsection{从入境城市乘坐列车到达亚琛}\label{subsec:从入境城市乘坐列车到达亚琛}

    从Deutsche Bahn (DB) 官网或其App (DB Navigator) 查询列车班次并线上购买车票即可 (需信用卡或 PayPal)。保存在个人移动设备上 (无需打印) 即可应对查票。火车站 DB 自助机或柜台也可直接购买车票,自助机一般不接受信用卡和 100欧以上的大额纸币。

    关于出发地,通常都会选择当地的主火车站 (Hauptbahnhof)。名字里带个Hauptbahnhof的是每个城市的大站,比较多列车班次。考虑到部分城市的机场和火车站的距离并不很近得靠机场巴士接应,并考虑到在机场取行李的时间,不建议在起飞前就订好车票。而且列车是班车,无需担心一天只有一班的情形。

    亚琛共四个火车站可作为目的地:Aachen Hbf,Westbahnhof,Schanzbahnhof,Rothe-Erde Bahnhof。可以选择离住所近且可以直达的火车站下车。通常Aachen Hbf是首选。

    法兰克福直达亚琛的车次一般为高速动车 (ICE),其车次较少。建议乘ICE至科隆后,在科隆转乘RE或RB1。从杜塞前往亚琛可以乘RE直达。

  \subsection{从车站乘公交到住所}\label{subsec:从车站乘公交到住所}

    可从AVV官网或其App (AVV connect)或上网查询线路。Google Map的线路足够可用。AVV也是在亚琛生活必不可少的app,除了线路之外可以用来查询公交到站时间。

    最直接的购票方式是在公交车上直接向司机购票,告诉他您要去的站即可,单程价格为2.7欧。也可以通过自动售票机 (Automaten) (常见于较大的公共车站),网上购票,客服中心 (KundenCentern) 购票。

    Tips:如果您的房东或认识的同学是在这读书的学生的话,不妨让ta来机场或车站接应:只要是北威州内的城市,在读学生凭学期票是免公共交通费的。

\section{重要事务办事点}\label{sec:重要事务办事点}

  在德国学习所涉及到重要事务主要有五个:Anmelden的市政厅 (B.6章,Fig.3)、入学注册的SuperC (B.7章,Fig.4)、延签的外管局 (B.8章,Fig.5)、在换住所时需要去更新个人信息的银行分行以及医保服务点。我们这里仅讨论前三个;银行和医保分别在B.1章和B.2章已有铺垫,这里不赘述。

  \subsection{市政厅}\label{subsec:市政厅}

    如Fig.3所示:实线-市政厅/Bürgerservice Katschhof;虚线-Rathaus Aachen/City Hall Aachen;点线-大教堂/Aachener Dom/Aachen Cathedral。

    通常留学生会用到的市政业务/Bürgerservice就是Anmelden;除此以外,当地人缴交交通违规罚款也算是Bürgerservice。亚琛负责Bürgerservice的有两个办事点:一个是Katschhof,即Fig.3的玻璃房子;另一个是Bahnhofplatz,在外管局/Ausländeramt (Fig.5) 大楼的其中一层。由于外管局以负责的外国人居留卡和签证业务为主,以Bürgerservice为辅。遂逐渐用市政厅唯一指代Bürgerservice Katschhof。

    Rathaus旧时候的确是亚琛的市政厅,但现在仅作观光之用,并不处理市政相关的事情,它英文名被称作City Hall仅仅是历史原因。

    亚琛唯一冠名 “大教堂” 的是点线圈起来的地标性建筑。亚琛的教堂有很多,有不少建筑长得就很像教堂,而且看起来都很大,譬如说Rathaus,它就在Aachener Dom对面且同样是地标性建筑,但Rathaus并不得名大教堂。

  \subsection{SuperC}\label{subsec:SuperC}

    A.2.2节谈到的外办(入学注册)和ZPA (考试)的办公室都在SuperC。SuperC前的大路叫Templergraben,如Fig.4中所示的直线。

  \subsection{外管局}\label{subsec:外管局}

    Fig.5左边的地标性建筑就是外管局/Ausländeramt,右边的地标性建筑就是亚琛主火/Aachen Hbf。外管局大楼除了负责外国人居留卡的业务以外还负责Anmelden即Bürgerservice业务,所以网上搜索Ausländeramt和Bürgerservice Bahnhofplatz会得到同一个地点。

\section{入境两周内必办事宜}\label{sec:入境两周内必办事宜}

  购买电话卡和办理宽带,签订租房合同以及Anmelden,银行卡激活,医保激活。这些都是您入境德国后要尽快落实的事情。有了Anmelden才能办理银行和医保的激活业务,您才可以使用德国的银行卡里的欧元进行消费,有了医保才能入学注册。

  每次更换地址 (找到了可Anmelden的房子) 后都要做类似的操作:Anmelden->银行+医保换地址,以免纸质信件丢失。

  \subsection{电话卡和宽带}\label{subsec:电话卡和宽带}

    德国的几家大型网络运营商是Vodafone、Telekom、o2、e-puls等等;更多选择可以从德国打折网查询。上网输入关键词搜索即可查询附近的服务点,购买电话卡和办理宽带。

    不同运营商的套餐和服务略有不同,常见的有prepaid和合同两类。简单解释是,前者可自由选择充钱与否,一般是从充钱之日算起一个月内有效,比较灵活;后者则类似于国内App的年付+续订,平均下来每个月比前者便宜。Vodafone有它对应的App叫Mein Vodafone,通过它可以线上充值,相对比较方便。

    所谓宽带是指租金里不含网费的、需要自行办理WiFi网络的那些住房。有些公司同时办理电话卡和宽带 (DSL) 会有折扣。总之,您可根据自己的需要比较各个公司之间的套餐,在办理的时候可以咨询一下店员。

  \subsection{签订住房合同以及Anmelden}\label{subsec:签订住房合同以及Anmelden}

    关于签订合同的注意事项已经在B.3章谈及过了。当您拿到租房合同之后最好在两周以内完成户口登记,然后分别去银行和医保门店换地址。

    Bürgerservice Katschhof麻雀虽小五脏俱全,排队的人通常不会很多;Bürgerservice Bahnhofplatz的业务则比较集中,除了Anmelden业务外还包括延签的居留卡的申请和领取。

    需要带上的文件通常为护照与住房合同。事前最好先问问房东 “我需要带哪些东西去Anmelden”:房东对Anmelden政策的了解会比本手册更清楚。

    去Anmelden时需要先在Bürgerservice大厅处拿号排队 (也可提前网上预约)。

    回家之后记得在自己的门牌上贴上自己的名字,或要求房东更换公寓楼下的住户列表,确保快递员和信使找得到您的名字。

  \subsection{银行账户激活}\label{subsec:银行账户激活}

    账户激活的事宜已分别就德银和Sparkasse Aachen讲过。有了Girokonto的IBAN后即可在没有收到银行卡的情况下签订房租、医保、电话卡和宽带的自动扣款合同。如果您更换了地址,则需要去亚琛的对应支行/Filiale 更改在银行注册的地址。

    备注:如B.1.1节,签证相关文件里有 “解冻”的字样。本小节的操作看起来像是 ”解冻”:那笔10,236欧被 “冻”在账号里不能用,通过上面的一系列操作使得卡里的钱能用,不就是“解冻”了吗? 注意区分 ”解冻”和 ”激活” 的概念:”解冻”在德国驻华使领馆->服务->签证和入境->长期停留->留学签证申请须知指的是如下情况:
    \begin{itemize}
      \item 未申请签证,因此也未赴德国;
      \item 已申请签证,但是已将该申请撤销;
      \item 已申请到签证,但未使用;
      \item 已申请到签证,但在获得外国人管理局签发的德国居留许可之前又离开了申根国家。
    \end{itemize}

    这些情况具有的共同点是,都没有在德国境内获得Anmelden证明,或说是都没有打算在德国境内长期居留,您想把您的被”冻结”在德国的银行账户里的10,236欧元从取出,这就称为“解冻”。而本节谈论的情况是,您已经计划在德国长期居留并且已经签了租房合同,希望使用卡里的钱作为日常消费:这叫 ”(账户)激活”,对应德文为Konto Freischaltung。

  \subsection{医保激活}\label{subsec:医保激活}

    关于医保,初次入境得解决两个事情:私保缴费和公保申请。

    直接在网上填写您的护照、住所和银行卡等信息即可办理保险;也可以去亚琛当地的TK或AOK门店通过工作人员办理。完成后会给您发确认邮件以及纸质信件,后者附带了医保证明 (用于入学注册) 以及医保卡。

    医保的扣款方式是自动扣款的。其实,有了Girokonto的IBAN码即可在没有收到银行卡的情况下签订自动扣款的合同。所以即使您还没有收到银行寄过来的卡,也可以完成本小节的所有步骤。

    如果您更换了地址,则需要去门店 (TK, AOK, …) 或对应官网更改在医保系统的地址。

\section{RWTH注册和管理系统}\label{sec:RWTH注册和管理系统}

  RWTH的 “注册” 分为入学注册和学期注册。在签订了公保合同以后,即可办理入学注册手续,与此同时您即可登录学校的学生管理系统管理您的各项事务。

  \subsection{入学注册}\label{subsec:入学注册}

    大学新生须按照录取通知书/Zulassung/Zu1上所写的时间到国际办公室/外办/International Office 注册。录取通知书上会指明注册需要的证件与材料,这其中通常包括 (请以录取通知书中的信息为准):
    \begin{enumerate}
      \item 护照
      \item 录取通知书
      \item 国内学校毕业证书的原件及翻译公证件
      \item 语言水平证书
      \item 医疗保险证明
      \item 若是转学,则需要之前学校的退学证明
      \item 如申请时未交登记照,则需要登记照一张
      \item 某些专业可能需要的GRE成绩
      \item Anmelden证明
    \end{enumerate}
    
    在外办完成注册后您会拿到外办开具的注册证明和一张银行转账单,转账注册费以完成注册。除此以外您还会拿到用于激活学生管理系统的Coupon,登录IdM Selfservice按提示激活您的系统,激活完成后您会看到自己的账号ab123456和8位字符 (大小写+数字) 的初始密码信息,并一定要记下来2。同时,上传一张个人照片以用于BlueCard,并及时更改您的住所信息。

    入学注册后您将会拿到学生证/蓝卡/BlueCard 和ASEAG的学期票/Semesterticket。BlueCard由学校有关部门通过邮件通知新生在规定时间内到SuperC前的广场领取。学期票则是邮寄到您的住所。

    在RWTH通行,您主要有两个号码:一个是蓝卡上显示的入学号/MATR.-NR. 12345,主要用于考试;另一个是用于登录学生管理系统的ab123456 (其中ab是您的名-姓缩写)。

  \subsection{学生管理系统}\label{subsec:学生管理系统}

    RWTH的学生管理系统几乎承包了这个学校的所有业务,如无意外几乎不必麻烦人工服务。

    RWTHonline (Fig.6) 负责的业务包含选课、报名考试\&查看成绩、打印在读证明\&成绩单 (其PDF档就已盖有学校公章)、语言中心报名、交学费等。

    IdM Selfservice (Fig.7) 可由RWTHonline登入,该系统主要用来管理您的账号密码以及您的设备在全校的WiFi服务:只要您身边有个RWTH的建筑,即可连上学校的eduroam的WiFi。

    RWTHmoodle (Fig.8) 管理所有学习资料,包括教授的讲义、提交作业和接收批改等等。

    RWTHApp (Fig.9) 则管理课程表、图书馆借书、食堂菜谱、空教室查询等等。

  \subsection{学生邮箱}\label{subsec:学生邮箱}

    注册不久,您就可以登录RWTHmail管理您的邮箱。邮箱格式为 vorname.nachname@rwth-aachen.de。假设您姓张名伟,则邮箱为 wei.zhang@rwth-aachen.de。

    但是要注意,登录RWTHmail所需的账号为ab123456@rwth-aachen.de,这里,ab123456是入学注册时您领到的学生管理系统的账号;登录RWTHmail的密码则是ab123456的对应密码。同样假设您的名字是张伟,则登录账号为 wz123456@rwth-aachen.de,而不是以 wei.zhang@rwth-aachen.de登录。

    通过这个邮箱,您可以享受绝大部分的学生优惠,包括GitHub的学生包。联系教授和申请实习的时候也尽量使用这个邮箱。

\section{延签和居留许可}\label{sec:延签和居留许可}

  国内申请到的入境签证为六个月的学生签证或为期三个月的的语言签证,都是短期签证。这个短期签证到期前一至两个月需要办理延签手续获得居住许可/居留卡/Aufenthaltstitel作为长期签证。延签分为两步:申请和取卡。

  \subsection{申请}\label{subsec:申请}

  第一步是申请。所需带去现场的材料如下:
  \begin{enumerate}
    \item 护照 (延签则另需要当前已有的居住许可)
    \item 证件照 (35mm*45mm)
    \item 在读证明 (Studienbescheinigung) 在RWTHonline首页Bescheinigungen (Dokumente) 下载打印
    \item 资金证明 (每月至少720欧,近几年增加至840欧) 或奖学金证明
    \item 医疗保险证明 (即保险公司处开具的Krankenversicherungsbescheinigung,延签也可用保险卡代替)
    \item 所需费用:第一次申请签证费用100欧,延签费用93欧。(获得德国奖学金的同学免除签证费)
  \end{enumerate}

  申请现场还需填写一张申请表。

  注意事项如下:

  \begin{itemize}
    \item 若您是读预科的同学,须直接去外管局办理申请业务。
    \item 对于读预科的同学,当您已经成为或即将成为RWTH的学生后,依然先去外管局办理延签申请。外管局的工作人员发现您已是RWTH的学生了,他们就会把您的档案转给RWTH的延签办公室,这样下一次您就可以直接去学校延签了。
    \item 若您是RWTH的学生,则前往负责RWTH的学生签证的办公室办理:Außenstelle des Ausländeramtes im Super C。
    \item 若您是FH Aachen及其他高校的学生,则前往Verwaltungsgebäude Hackländerstraße 1办理。
    \item 若之前在德国其他城市申请过签证或有过延签,则第一次在亚琛办理改签/延签也需要在外管局办理。
    \item 原则上申根签证是在获得居住许可之后才会取得。所以,仅持入境签证 (即本节开头的学生签证和语言签证) 是不可以去周边国家 (旅行) 的。
    \item 在外管局办理延签需要提前预约,在SuperC办理延签则是以排队取号的形式。在Super C办理延签要特别注意避开签证高峰期,否则需要早上五六点去门口排队,而且亚琛早上没有商店和自习室开门,切记多穿一点。
  \end{itemize}

  \subsection{取卡}\label{subsec:取卡}

  第二步是取卡。领取居留许可则是等纸质信件的通知去外管局领取,B.5.3节的Fig.5展示了外管局与亚琛主火的相对位置。领取的大厅同样是领号的,因为办理事务的人较多,一般需要等待一个小时左右。
