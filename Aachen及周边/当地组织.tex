%!TEX root = ../hcwsa.tex

\chapter{当地组织}

\section{中国驻杜塞尔多夫总领馆}

中国驻杜塞尔多夫总领事馆作为中国在德国新设外交机构,其主要任务是大力促进中国与德国特别是总领馆领区的政治、经济、人文等各领域交流合作,切实维护中国公民和企业的安全与合法权益,千方百计为领区华侨华人在当地的长期生存和发展创造便利条件,全心全意为领区各界对华往来提供优质服务。

领事馆微信公众号:dusinfo

领事馆负责护照补办以及签发等业务。具体办理流程可点击链接到官网查询。

\begin{figure}[h]
\centering
\includegraphics[width=.5\textwidth]{Aachen及周边/Düsseldorf.png}
\end{figure}

\section{RWTH 的职能部门}

这里列出在RWTH学习和生活,通常需要求助的职能部门。所列之问题是包括但不限于的。

\href{https://www9.rwth-aachen.de/go/id/xga/}{\textsbf{亚琛工业大学中文网站}}:关于RWTH的介绍以及申请到RWTH就读本科、硕士和博士的一些基本条件,各个专业设置的介绍,以及北京代表处的联系方式。

\href{https://www.rwth-aachen.de/go/id/pvd/lidx/1}{\textsbf{国际办公室/外办/International Office}}:主要负责协调RWTH在国际上的关系、负责国际代表团的访问活动,致力于与外国伙伴学校发展合作项目,包括给予一些想要在RWTH学习并毕业的同学一些建议,如录取和新生入学注册。
ZPA/Abteilung Zentrales Prüfungsamt:在结构上是RWTH管理学院学术和学生事务部的组成部分,是学生和考官就考试事宜进行联系的中心。其职能包括但不限于:管理考试的注册和退出、注册成绩、签发证书、审查考试或论文的批准与否,等等,适用于所有相关人员,包括教职员工和学生。

各学院专业的Studienberater (网上搜索RWTH+专业名+Studienberater):专业方面的咨询,如申请硕士时候的课程匹配度、研究方向的选择、毕业论文、移除不良成绩,等等。
Zentrale Studienberatung:RWTH主页->Studium->Beratung \& Hilfe:提供各类咨询,包括专业选择咨询、换专业咨询、心理咨询、职业生涯咨询等。

\section{亚琛中国学生学者联合会}

\href{http://www.vcwsa.rwth-aachen.de/}{\textsbf{亚琛学生学者联合会/Verein der Chinesischen Wissenschaftler und Studenten in Aachen .e.V.}}是成立于20世纪80年代、由中国学生及访问学者组成,由中国大使馆、亚琛工业大学留学生办事处认可并支持的非盈利性的公益组织。

联合会的宗旨是:帮助亚琛及周边地区追求卓越的学生学者们,为他们的个人发展提供舞台,为他们的丰富生活创造条件。同时通过我们的努力,广泛联络亚琛华人的力量,在本地区介绍和宏扬中国文化,扩大影响,推进中德人民和中外青年学子之间的友谊。让亚琛地区的国际友人体验中国文化的同时也让在亚琛的中国学生了解和体验德国文化做好中德之间的桥梁。

亚琛学联官方微信帐号:RWTHChina

\section{亚琛中德协会}

\href{https://www.cg-society.com/en/}{\textsbf{亚琛中德协会/Chinese-German Society Aachen e.V.}}成立于2017年,是由亚琛的中德两国学生共同创办的公益学生协会。自创立起,协会致力于增进中德学生的相互了解,开辟并发展中德学生从生活到文化的多层次交流渠道,建立跨国界的友谊。该协会定期举办各色社交类和学术类的活动,有成员聚会、烹饪、远足和庆祝中德传统节日的活动,也有讲座、研讨会及企业探访等,以帮助新到亚琛的学生更好地适应在德国的学习和生活,也为准备去中国交换或读双学位的学生提供指导和帮助。

协会的自我定位为学生、学校及所有对中德交流感兴趣的人之间的桥梁,并为此推出了种类繁多的交流活动。

\section{各类学生社团}

亚琛当地有不少中国学生组建的社团,并一直欢迎新成员的加入。这些社团包括但不限于:

\textsbf{Over Se7en 乐队},成立于2019年10月,是一支由亚琛工大学生组成的摇滚乐队。乐队以Cover为主,同时也在做原创摇滚。自成立以来,Over Se7en参加了亚琛好声音开场以及中场演出,亚琛春晚演出,以及在亚琛举办摇滚专场live。

\textsbf{亚琛弦乐队colours}是一支即将成立的以弦乐重奏为主的乐队。colours成立后,可以和Over Se7en摇滚乐队合作前卫摇滚以及发挥想象力对于歌曲以及古典曲目做改编,做出自己的创作和风采。

\textsbf{汉服社}不仅以汉服为载体,弘扬中华民族传统文化;更是以汉服为非语言符号,通过踏青、马术节表演和亚琛春晚舞台实现中西文化的碰撞。

\textsbf{亚琛民乐队}是一支由二胡、古筝、琵琶、笛等民族乐器共同组成的队伍,其成员全部是非音乐专业的在校学生,但却是一支具有较高专业素养和崇高艺术追求的团队。在这两年里,亚琛民乐队先后参加了亚琛、科隆、波鸿学联组办春节晚会、北威州青年文艺汇演、科隆“中国节”,并且还受邀参加德国越南友好协会的春晚表演。

上述社团的具体介绍和风采照片皆可在\href{http://www.vcwsa.rwth-aachen.de/category/lives/社团风采/}{学联官网}上找到。如欲加入可注明联系目的发送邮件至\href{mailto:vcwsa.register@gmail.com}{vcwsa.register@gmail.com}。
