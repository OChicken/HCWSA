%!TEX root = ../hcwsa.tex

\chapter{亚琛概况}

\section{亚琛市}

\href{https://www.aachen.de/CHIN/kurzinfo.html}{\textsbf{亚琛/Aachen}}\footnote{亚琛市网站有汉语介绍。} ,是德国\href{https://www.land.nrw/en/welcome}{\textsbf{北莱茵-威斯特法伦州/北威州/Nordrhein-Westfalen/North Rhine-Westphalia}}\footnote{北威州网站有英语、德语和荷兰语三种语言介绍。}的一个边陲城市,她位于欧洲的中心,地处德国、荷兰和比利时三国交界处。因为地理位置的原因,亚琛也是一个非常重要的铁路汇集点,同时也是工业汇集中心,素有 ”欧洲心脏”之称。

亚琛拥有众多大学和科研机构,其中最负盛名的学校当属亚琛工业大学。亚琛工业大学吸引了约三万一千名学子来此学习深造。通过校方和当地学者的努力,亚琛在过去的几年中顺利完成了地区结构转型。汽车工程、激光技术、微芯片结构及医学领域的研究,让这个曾经的矿工业区变成了如今的高科技基地。高等专科学校和音乐学院也让大学生的总人数上升至四万人。与此同时,还有很多参加专业会议、学术交流的专家和学者经常来到亚琛。欧洲国会中心也为到访者提供了多功能的会议场所。年轻人的加入给这座古老的城市注入了生机和年轻的血液,也它也焕发出青春的活力。

\section{亚琛工业大学}

\href{https://www.rwth-aachen.de/go/id/a/?lidx=1}{\textsbf{亚琛工业大学/Rheinisch-Westfälische Technische Hochschule Aachen/RWTH Aachen}},创建于1870年,是位于德国亚琛的世界顶尖理工类大学,世界百强大学。其在工科领域享有极高的声誉,是欧洲顶尖理工大学IDEA联盟成员之一。

亚琛工业大学现为11所德国精英大学 (Exzellenzinitiative) 之一、9所顶尖德国理工大学联盟 (TU9) 成员之一。其同时是TIME欧洲顶尖工业管理者高校联盟、CESAER欧洲高等工程教育和研究大学会议联盟、PEGASUS欧洲航空航天大学联盟等一系列组织成员。

顶尖的科研与教育水平让许多著名公司如微软、福特、爱立信、飞利浦、联合技术等都在亚琛建立了分部,三菱则在附近建立了欧洲半导体中心。亚琛工业大学的校友出众,学术界有钱学森的导师冯·卡门,工业界有西门子、保时捷、宝马、奥迪、宾利等企业总裁。前中国科学院院长路甬祥、教育部副部长韦钰、原清华大学校长王大中也毕业于该校。

学校自创建以来产生过6位诺贝尔奖得主,10位莱布尼茨奖得主。学校尤其重视国际化合作,是从工业和企业界获得最多经费的德国大学。

\section{亚琛应用科技大学}

\href{https://www.fh-aachen.de/}{\textsbf{亚琛应用科技大学/Fachhochschulen Aachen - Aachen University of Applied Sciences/FH Aachen}}成立于1971年,是德国著名的应用科技大学之一。该校由多所应用技术大学和职业培训中心合并而成,100多年来落实以实践为导向的教育传统,在电气、机械工程、信息学等应用科学领域名列德国第一。

在研究方面,亚琛应用科技大学力争成为德国最强大的应用技术大学之一。能力主要是在未来的能源、移动和生命科学领域。最新的研究成果直接纳入教学。其机械工程和机电一体化在全德同类大学 (应用科技大学或高等专科学校) 中排名第一位。

\section{亚琛语言学院}

在德语作为外语 (DaF) 领域,\href{https://www.spraachen.org/}{\textsbf{亚琛语言学院}}作为地区最大的语言学院,开设全面的强化及备考课程,涵盖德语水平等级A1至C1,为每年不断增多的有学术背景的德语学习者服务:大学申请人,大学生,硕博研究生,已完成培训的学者以及在职人员是该校课程主要面向的群体。作为官方认证的考试中心,该校定期举办各类标准化语言考试。

几年前分布亚琛各地的亚琛语言学院各专业部门于2011年一起搬进坐落于亚琛市中心的Haus der Kohle。亚琛语言学院以德国为中心现设有以下三处办公地点:亚琛,于利希,以及北京 (中国分部)。

\section{于利希研究中心}

\href{https://www.fz-juelich.de/portal/DE/Home/home_node.html}{\textsbf{于利希研究中心/Jülich Forschungszentrum/FZJ}}是德国亥姆霍兹国家研究中心联合会的下属科研机构,主要从事物质结构、能源、信息、生命、环境和运输航天等方向的研究。现有超过5000名研究人员,是欧洲最大的科学研究机构之一。

研究中心在核物理、磁共振脑成像、太阳能电池和高倍透射电镜等方面的研究处于世界前沿。其中固体研究所Peter Grünberg 教授因发现巨磁电阻效应而荣获2007年诺贝尔物理学奖。
于利希研究中心与亚琛工业大学成立了\href{https://www.jara.org/en/}{\textsbf{Jülich-Aachen研究联盟/JARA}},多年来吸引了RWTH的众多毕业生到FZJ从事博士课题研究。
