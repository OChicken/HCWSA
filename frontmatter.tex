%!TEX root = hcwsa.tex

\chapter*{制作团队}

\addcontentsline{toc}{chapter}{制作团队}

  \textbf{编委}:

  \href{https://OChicken.net/}{马守然}\qquad 余宛鸿

  \vskip\baselineskip

  \textbf{责任编辑}:
  
  艾迪雅    \qquad 李俊东\qquad 吕天宇\qquad \href{https://OChicken.net/}{马守然}\qquad 秦瑞楷\qquad 孙煜林
  
  王\quad 博\qquad\ 王佳艺\qquad 王雨晴\qquad 杨笑宇\qquad 赵世豪\qquad 张益臣

  \vskip\baselineskip

  \textbf{鸣谢}:

  王春燕\qquad 张\quad 瑞

  \vskip\baselineskip

  \copyright 版权所有:\href{http://www.vcwsa.rwth-aachen.de/}{亚琛中国学生学者联合会}

  \newpage

\chapter*{序言}

\addcontentsline{toc}{chapter}{序言}

  亚琛,这座位于德国西部的边陲小镇,拥有着世界著名的高等学府。年复一年,一批又一批来自中国的莘莘学子抱着对未知的向往与人生的梦想,不远万里来到亚琛开始了他们的留学生涯。对于出门在外的留学生而言,中国与德国,这两种截然不同的文化与社会风俗,带来的不仅仅是心理上的冲击,更多的是初到陌生环境的茫然和无助。为了能在新同学们初到亚琛的时候就给予他们帮助,使他们能够克服最开始也是最艰巨的障碍,亚琛学生会的前辈们一起编写了这本亚琛新生手册,从衣食住行各个方面为新生们进行了细致的指导。在手册出版过后的这些年里,有无数的新生,包括作为新版编者的我们因此而受益。时光飞逝,岁月如梭,2012版、2017版和2020版的《亚琛新生手册》的出版已过去多年了,当我们再次翻开上版手册时,依旧能感受到前辈们对这项公益工作倾注的热忱和心血。不过在这些年,亚琛的人和物也发生着巨大的变化,新生事物在不断产生,过去的经验慢慢不能适应新的形势。这激励我们秉承前辈们的理念,拿起笔头更新和出版2023版《亚琛新生手册》,为广大学子提供更新更全的信息总汇。

  在新版的手册中,我们先集中收集和整理了从国内出发以及初到亚琛大概两周之内所需要的基本信息,主要包括交通信息,户口注册和延长签证,银行开户以及办理保险等。然后介绍了包括吃喝玩乐在内的亚琛学习与生活的种种。我们希望此新版手册能帮助即将来到亚琛求学、工作和生活的同胞更快更易地融入到德国的生活和学习中。

  此版的信息和稿件主要来自参与编写工作的学生会成员的自己亲身经历、网络以及其他渠道收集和整理的实用信息。 此外,我们也借鉴了部分《亚琛公派学联新生手册》2019版中的内容,在此我们也要感谢亚琛前辈们为此付出的辛勤劳动。

  在此感谢所有参与此版编撰工作的同学们,感谢大家利用自己的课余时间积极的参与到这项公益工作中。之后我们将继续完善和更新这个手册,使其内容更加丰富。

  \rightline{亚琛中国学生学者联合会}

  \rightline{2023年4月}

  \newpage

\chapter*{修订}

\addcontentsline{toc}{chapter}{修订}

  作为第四版的新生手册,相比之前的版本,我们加入、重构以及删节了不少内容,使得新生手册更与时俱进。

  相较于往期的新生手册,我们并没有给出办公组织、餐馆、超市等场所的具体地址或营业时间等信息,而是给出其全称,用\textsbf{无衬线加粗}字体将它们强调,并提供了相关网址的超链接。读者们可以借助网址链接、搜索关键词、Google Map等工具自行找到更为确切的地址、营业时间、照片等相关信息。

  RWTH有非常完善的学生管理系统。我们也提供了该系统的截图 (已用虚构信息替代了敏感信息),便于读者对在RWTH的学习有更具体的认识。

  在构思本手册的初期,我们考虑过如下内容,但最终都并没有加入。如果有不少读者反馈说需要这些内容,我们将在下一版本加入。

  Sec \ref{sec:交通} - 国内驾照转德国驾照。即使国内驾照是可以在德国租车和开车,但是若停留超过六个月则依然要使用德国的驾照。这就会涉及到如何选驾校、如何驾校考试、如何租车、如何买车等一系列的问题,面向的读者群就少很多。其实关于这个问题可以在网上找到不少攻略,因此我们并不加入自驾车的内容。

  Sec \ref{sec:交通} - 长途大巴。长途大巴可以作为城际出行的一个选择,但是这种交通工具并不被大多数同学采用,毕竟我们有学期票,票价上的压力并不很大,所以也并没有加入。

  Sec \ref{sec:生活的一切琐碎问题} - 防骗。其实只要我们怀有适当的防人之心,在遇到涉及金额较多的交易中保持警惕,并根据自己在当地的人脉网多方求证,即可过滤掉绝大多数的骗局。因此不打算在此详述。
