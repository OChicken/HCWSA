%!TEX root = hcwsa.tex

\chapter{Aachen 及周边}\label{chap:Aachen 及周边}

\section{亚琛概况}\label{sec:亚琛概况}

  \subsection{亚琛市}\label{subsec:亚琛市}

    \href{https://www.aachen.de/CHIN/kurzinfo.html}{\textsbf{亚琛/Aachen}}\footnote{亚琛市网站有汉语介绍。} ,是德国\href{https://www.land.nrw}{\textsbf{北莱茵-威斯特法伦州/北威州/Nordrhein-Westfalen/North Rhine-Westphalia}}\footnote{北威州网站有英语、德语和荷兰语三种语言介绍。}的一个边陲城市,她位于欧洲的中心,地处德国、荷兰和比利时三国交界处。因为地理位置的原因,亚琛也是一个非常重要的铁路汇集点,同时也是工业汇集中心,素有 ”欧洲心脏”之称。

    亚琛拥有众多大学和科研机构,其中最负盛名的学校当属亚琛工业大学。亚琛工业大学吸引了约三万一千名学子来此学习深造。通过校方和当地学者的努力,亚琛在过去的几年中顺利完成了地区结构转型。汽车工程、激光技术、微芯片结构及医学领域的研究,让这个曾经的矿工业区变成了如今的高科技基地。高等专科学校和音乐学院也让大学生的总人数上升至四万人。与此同时,还有很多参加专业会议、学术交流的专家和学者经常来到亚琛。欧洲国会中心也为到访者提供了多功能的会议场所。年轻人的加入给这座古老的城市注入了生机和年轻的血液,也它也焕发出青春的活力。

  \subsection{亚琛工业大学}\label{subsec:亚琛工业大学}

    \href{https://www.rwth-aachen.de/go/id/a/?lidx=1}{\textsbf{亚琛工业大学/Rheinisch-Westfälische Technische Hochschule Aachen/RWTH Aachen}},创建于1870年,是位于德国亚琛的世界顶尖理工类大学,世界百强大学。其在工科领域享有极高的声誉,是欧洲顶尖理工大学IDEA联盟成员之一。

    亚琛工业大学现为11所德国精英大学 (Exzellenzinitiative) 之一、9所顶尖德国理工大学联盟 (TU9) 成员之一。其同时是TIME欧洲顶尖工业管理者高校联盟、CESAER欧洲高等工程教育和研究大学会议联盟、PEGASUS欧洲航空航天大学联盟等一系列组织成员。

    顶尖的科研与教育水平让许多著名公司如微软、福特、爱立信、飞利浦、联合技术等都在亚琛建立了分部,三菱则在附近建立了欧洲半导体中心。亚琛工业大学的校友出众,学术界有钱学森的导师冯·卡门,工业界有西门子、保时捷、宝马、奥迪、宾利等企业总裁。前中国科学院院长路甬祥、教育部副部长韦钰、原清华大学校长王大中也毕业于该校。

    学校自创建以来产生过6位诺贝尔奖得主,10位莱布尼茨奖得主。学校尤其重视国际化合作,是从工业和企业界获得最多经费的德国大学。

  \subsection{亚琛应用科技大学}\label{subsec:亚琛应用科技大学}

    \href{https://www.fh-aachen.de/}{\textsbf{亚琛应用科技大学/Fachhochschulen Aachen - Aachen University of Applied Sciences/FH Aachen}}成立于1971年,是德国著名的应用科技大学之一。该校由多所应用技术大学和职业培训中心合并而成,100多年来落实以实践为导向的教育传统,在电气、机械工程、信息学等应用科学领域名列德国第一。

    在研究方面,亚琛应用科技大学力争成为德国最强大的应用技术大学之一。能力主要是在未来的能源、移动和生命科学领域。最新的研究成果直接纳入教学。其机械工程和机电一体化在全德同类大学 (应用科技大学或高等专科学校) 中排名第一位。

  \subsection{亚琛语言学院}\label{subsec:亚琛语言学院}

    在德语作为外语 (DaF) 领域,\href{https://www.spraachen.org/}{\textsbf{亚琛语言学院}}作为地区最大的语言学院,开设全面的强化及备考课程,涵盖德语水平等级A1至C1,为每年不断增多的有学术背景的德语学习者服务:大学申请人,大学生,硕博研究生,已完成培训的学者以及在职人员是该校课程主要面向的群体。作为官方认证的考试中心,该校定期举办各类标准化语言考试。

    几年前分布亚琛各地的亚琛语言学院各专业部门于2011年一起搬进坐落于亚琛市中心的Haus der Kohle。亚琛语言学院以德国为中心现设有以下三处办公地点:亚琛,于利希,以及北京 (中国分部)。

  \subsection{于利希研究中心}\label{subsec:于利希研究中心}

    \href{https://www.fz-juelich.de/portal/DE/Home/home_node.html}{\textsbf{于利希研究中心/Jülich Forschungszentrum/FZJ}}是德国亥姆霍兹国家研究中心联合会的下属科研机构,主要从事物质结构、能源、信息、生命、环境和运输航天等方向的研究。现有超过5000名研究人员,是欧洲最大的科学研究机构之一。

    研究中心在核物理、磁共振脑成像、太阳能电池和高倍透射电镜等方面的研究处于世界前沿。其中固体研究所Peter Grünberg 教授因发现巨磁电阻效应而荣获2007年诺贝尔物理学奖。

    于利希研究中心与亚琛工业大学成立了\href{https://www.jara.org/en/}{\textsbf{Jülich-Aachen研究联盟/JARA}},多年来吸引了RWTH的众多毕业生到FZJ从事博士课题研究。

\section{当地组织}\label{sec:当地组织}

  \subsection{中国驻杜塞尔多夫总领馆}\label{subsec:中国驻杜塞尔多夫总领馆}

    \href{http://dusseldorf.china-consulate.org/chn/}{\textsbf{中国驻杜塞尔多夫总领事馆}}作为中国在德国新设外交机构,其主要任务是大力促进中国与德国特别是总领馆领区的政治、经济、人文等各领域交流合作,切实维护中国公民和企业的安全与合法权益,千方百计为领区华侨华人在当地的长期生存和发展创造便利条件,全心全意为领区各界对华往来提供优质服务。

    领事馆微信公众号:dusinfo

    领事馆负责\textbf{护照补办以及签发}等业务。具体办理流程可点击链接到官网查询。

    \begin{figure}[h]
      \centering
      \vskip-.5\baselineskip
      \includegraphics[width=.5\textwidth]{Aachen及周边/Düsseldorf.png}
      %\vspace{-\baselineskip}
      \vskip-2\baselineskip
    \end{figure}

  \subsection{RWTH 的职能部门}\label{subsec:RWTH 的职能部门}

    这里列出在RWTH学习和生活,通常需要求助的职能部门。所列之问题是包括但不限于的。

    \href{https://www.beijing.rwth-aachen.de/go/id/xga/}{\textsbf{亚琛工业大学中文网站}}:关于RWTH的介绍以及申请到RWTH就读本科、硕士和博士的一些基本条件,各个专业设置的介绍,以及北京代表处的联系方式。

    \href{https://www.rwth-aachen.de/go/id/pvd/lidx/1}{\textsbf{国际办公室/外办/International Office}}:主要负责协调RWTH在国际上的关系、负责国际代表团的访问活动,致力于与外国伙伴学校发展合作项目,包括给予一些想要在RWTH学习并毕业的同学一些建议,如录取和新生入学注册。

    \href{https://www.rwth-aachen.de/cms/root/Die-RWTH/Einrichtungen/Verwaltung/Dezernate/Akademische-und-studentische-Angelegenhe/~rcv/Abteilung-1-3-Zentrales-Pruefungsamt/}{\textsbf{ZPA/Abteilung Zentrales Prüfungsamt}}:在结构上是RWTH管理学院学术和学生事务部的组成部分,是学生和考官就考试事宜进行联系的中心。其职能包括但不限于:管理考试的注册和退出、注册成绩、提交论文、签发证书、审查考试或论文的批准与否,等等,适用于所有相关人员,包括教职员工和学生。

    \href{https://www.rwth-aachen.de/cms/root/studium/Vor-dem-Studium/Internationale-Studierende/Organisation-des-Studienaufenthaltes/Visum-Aufenthaltsrecht/~bpte/Aufenthaltserlaubnis/}{\textsbf{Außenstelle des Ausländeramts im SuperC}}: \textsbf{外管局/Ausländeramt} (Sec \ref{subsec:外管局}) 在 SuperC 的办公室,负责 RWTH 学生的居留许可的认证。

    \textbf{各学院专业的Studienberater} (网上搜索\textbf{RWTH+专业名+Studienberater}):专业方面的咨询,如申请硕士时候的课程匹配度、研究方向的选择、毕业论文、移除不良成绩,等等。

    \textsbf{Zentrale Studienberatung}:RWTH主页->Studium->\href{https://www.rwth-aachen.de/cms/root/Studium/~hzvj/Beratung-Hilfe/}{\textsbf{Beratung \& Hilfe}}:提供各类咨询,包括专业选择咨询、换专业咨询、心理咨询、职业生涯咨询等。

  \subsection{亚琛中国学生学者联合会}\label{subsec:亚琛中国学生学者联合会}

    \href{http://www.vcwsa.rwth-aachen.de/}{\textsbf{亚琛学生学者联合会/Verein der Chinesischen Wissenschaftler und Studenten in Aachen .e.V.}}是成立于20世纪80年代、由中国学生及访问学者组成,由中国大使馆、亚琛工业大学留学生办事处认可并支持的非盈利性的公益组织。

    联合会的宗旨是:帮助亚琛及周边地区追求卓越的学生学者们,为他们的个人发展提供舞台,为他们的丰富生活创造条件。同时通过我们的努力,广泛联络亚琛华人的力量,在本地区介绍和宏扬中国文化,扩大影响,推进中德人民和中外青年学子之间的友谊。让亚琛地区的国际友人体验中国文化的同时也让在亚琛的中国学生了解和体验德国文化做好中德之间的桥梁。

    亚琛学联微信公众号:vcwsaachen。如需加入\textbf{新生群}可添加小助手账号:RWTHChina。

    \begin{figure}[h]
    \centering
    \includegraphics[width=.4\textwidth]{Aachen及周边/vcwsaachen.jpg}
    \vskip-\baselineskip
    \end{figure}

  \subsection{亚琛中德协会}\label{subsec:亚琛中德协会}

    \href{https://www.cg-society.com/en/}{\textsbf{亚琛中德协会/Chinese-German Society Aachen e.V.}}成立于2017年,是由亚琛的中德两国学生共同创办的公益学生协会。自创立起,协会致力于增进中德学生的相互了解,开辟并发展中德学生从生活到文化的多层次交流渠道,建立跨国界的友谊。该协会定期举办各色社交类和学术类的活动,有成员聚会、烹饪、远足和庆祝中德传统节日的活动,也有讲座、研讨会及企业探访等,以帮助新到亚琛的学生更好地适应在德国的学习和生活,也为准备去中国交换或读双学位的学生提供指导和帮助。

    协会的自我定位为学生、学校及所有对中德交流感兴趣的人之间的桥梁,并为此推出种类繁多的交流活动。

  \subsection{各类学生社团}\label{subsec:各类学生社团}

    亚琛当地有不少中国学生组建的\href{http://www.vcwsa.rwth-aachen.de/category/lives/社团风采/}{社团},并一直欢迎新成员的加入。这些社团包括但不限于:

    \textsbf{Over Se7en 乐队},成立于2019年10月,是一支由亚琛工大学生组成的摇滚乐队。乐队以Cover为主,同时也在做原创摇滚。自成立以来,Over Se7en参加了亚琛好声音开场以及中场演出,亚琛春晚演出,以及在亚琛举办摇滚专场live。

    \textsbf{亚琛弦乐队colours}是一支即将成立的以弦乐重奏为主的乐队。colours成立后,可以和Over Se7en摇滚乐队合作前卫摇滚以及发挥想象力对于歌曲以及古典曲目做改编,做出自己的创作和风采。

    \textsbf{汉服社}不仅以汉服为载体,弘扬中华民族传统文化;更是以汉服为非语言符号,通过踏青、马术节表演和亚琛春晚舞台实现中西文化的碰撞。

    \textsbf{亚琛民乐队}是一支由二胡、古筝、琵琶、笛等民族乐器共同组成的队伍,其成员全部是非音乐专业的在校学生,但却是一支具有较高专业素养和崇高艺术追求的团队。在这两年里,亚琛民乐队先后参加了亚琛、科隆、波鸿学联组办春节晚会、北威州青年文艺汇演、科隆``中国节'',并且还受邀参加德国越南友好协会的春晚表演。

    如需加入可注明联系目的发送邮件至\href{mailto:vcwsa.register@gmail.com}{vcwsa.register@gmail.com}。
