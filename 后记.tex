%!TEX root = hcwsa.tex

\chapter*{后记}

\addcontentsline{toc}{chapter}{后记}

  虽然我们总说 “德村” ”亚村”,但她的基建还是能满足现代社会的基本需求的,譬如WiFi,网络,(通过德银/Sparkasse Aachen等) 线上支付,libgen上的电子书随便下,跟国内并没有太大差别,而且物价还十分友好;手机、pad、电脑等电子产品并不算是基本需求,并且还有知识产权保护,所以相比国内贵 (但这些可以回国买 (大雾))。

  排队排很久 (如Anmelden或者去银行) 的事情,国内的银行也排很久;电话卡合同一不小心又续一年的情形,国内的App也是类似情况;有对外国人不友好/英语不流利的工作人员,国内完全不懂英文的银行/餐厅工作人员也大有人在;在一门课上,正在学德语的本小编遇到一个正在学习仓颉输入法的德国同学然后他们成为了好朋友——国家认同和民族主义是人之常情,但不要总跟 “狭隘” 二字扯上关系。
