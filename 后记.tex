%!TEX root = hcwsa.tex

\chapter*{后记}

\addcontentsline{toc}{chapter}{后记}

  作为第三版的新生手册,相比之前的版本,我们加入、重构以及删节了不少内容,使得新生手册更与时俱进。在构思本手册的初期,我们考虑过如下内容,但最终都并没有加入。如果有不少读者反馈说需要这些内容,我们将在下一版本加入。

  C.2 交通- 国内驾照转德国驾照。即使国内驾照是可以在德国租车和开车,但是若停留超过六个月则依然要使用德国的驾照。这就会涉及到如何选驾校、如何驾校考试、如何租车、如何买车等一系列的问题,面向的读者群就少很多。其实关于这个问题可以在网上找到不少攻略,因此我们并不加入自驾车的内容。

  C.2 交通-长途大巴。长途大巴可以作为城际出行的一个选择,但是这种交通工具并不被大多数同学采用,毕竟我们有学期票,票价上的压力并不很大,所以也并没有加入。

  C.3 日常消费和饮食-换汇。在国外,换汇是个很常见的需求,有欧元换人民币,也有人民币换欧元,目的不尽相同。网上能够搜到不少提供换汇服务的网站,微信群里也有不少人经常发广告。为避免误导,我们不在此手册里提供信息源。

  C.5 生活的一切琐碎问题-防骗。其实只要我们怀有适当的防人之心,在遇到涉及金额较多的交易中保持警惕,并根据自己在当地的人脉网多方求证,即可过滤掉绝大多数的骗局。因此不打算在此详述。

  虽然我们总说 “德村” ”亚村”,但她的基建还是能满足现代社会的基本需求的,譬如WiFi,网络,(通过德银/Sparkasse Aachen等) 线上支付,libgen上的电子书随便下,跟国内并没有太大差别,而且物价还十分友好;手机、pad、电脑等电子产品并不算是基本需求,并且还有知识产权保护,所以相比国内贵 (但这些可以回国买 (大雾))。

  排队排很久 (如Anmelden或者去银行) 的事情,国内的银行也排很久;电话卡合同一不小心又续一年的情形,国内的App也是类似情况;有对外国人不友好/英语不流利的工作人员,国内完全不懂英文的银行/餐厅工作人员也大有人在;在一门课上,正在学德语的本小编遇到一个正在学习仓颉输入法的德国同学然后他们成为了好朋友——国家认同和民族主义是人之常情,但不要总跟 “狭隘” 二字扯上关系。
